
% From https://www.overleaf.com/learn/latex/Glossaries

\makeglossaries % Prepare for adding glossary entries


\newglossaryentry{latex}
{
        name=latex,
        description={Is a mark up language specially suited for
scientific documents}
}

\newglossaryentry{bibliography}
{
        name=bibliography,
        plural=bibliographies,
        description={A list of the books referred to in a scholarly work,
typically printed as an appendix}
}

\newglossaryentry{maths}
{
    name=mathematics,
    description={Mathematics is what mathematicians do}
}

\newglossaryentry{technical_debt}{
    name={technical debt}, 
    description={Software engineering term for the degradation of code quality that occurs over time as a software system is developed and evolved}
    }

% --------------------
% ----- Acronyms -----
% --------------------

\newacronym{phd}{PhD}{philosophiae doctor}
\newacronym{CoPCSE}{CoPCSE@NTNU}{Community of Practice in Computer ScienceEducation at NTNU}
\newacronym{gcd}{GCD}{Greatest Common Divisor}
\newacronym{slr}{SLR}{systematic literature review}
\newacronym{sms}{SMS}{systematic mapping study}
\newacronym{mlr}{MLR}{multivocal literature review}
\newacronym{gl}{GL}{grey literature}
\newacronym{wl}{WL}{white literature}
\newacronym{se}{SE}{Software Engineering}
\newacronym{ml}{ML}{Machine Learning}
\newacronym{ai}{AI}{Artificial Intelligence}
\newacronym{ci}{CI}{continuous integration}
\newacronym{cd}{CD}{continuous deployment}
\newacronym{cde}{CDE}{continuous delivery}
\newacronym{ka}{KA}{knowledge area}
