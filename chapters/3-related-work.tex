\chapter{Related Work}
\label{ch:related_work}

\textcite{Shahin2017} performed a comprehensive \acrfull{slr} of tools, practices, approaches and challenges for continuous software engineering.
The authors identified thirty approaches and associated tools for facilitating continuous software engineering practices, and found they were being applied in a wide variety of application domains.
The application of the identified approaches to \acrshort{ml} systems is not discussed or mentioned.

\textcite{Rodriguez2017} performed a \acrfull{sms} of \acrfull{cd}, focusing on the state of research, characteristics of \acrshort{cd} and benefits, challenges and gaps in the research.
The authors identify nine concrete areas for future research opportunities, but the topic of applying \acrshort{cd} to \acrshort{ml} is not discussed.

\textcite{Baier2019} surveyed the academic literature for challenges in deploying \acrshort{ml} models.
The authors also conducted eleven semi-structured expert interview with \acrshort{ml} practitioners from a wide variety of industries.
The data from the literature survey and expert interviews were combined and analyzed in order to produce an overview of practical challenges surrounding \acrshort{ml} within the categories of pre-deployment, deployment, and non-technical issues.
The authors found that the practitioners' experiences generally confirmed challenges reported in the academic literature.

\textcite{Paleyes2020} did a survey of case studies on deploying \acrshort{ml} models with the goal of outlining a research agenda for addressing the challenges that practitioners were found to encounter.
The authors identified practical challenges in sixteen different steps of \acrshort{ml} deployment, and suggest further research to be done in the areas of tooling and services for individual challenges, as well as new holistic approaches for dealing with \acrshort{ml} systems engineering.

\textcite{Kumeno2020} performed an \acrshort{slr} of \acrshort{se} challenges for \acrshort{ml}, and mapped them to the twelve \acrfullpl{ka} of the SWEBOK \cite{Bourque2014}.
The author found that challenges in \acrshort{se} for \acrshort{ml} are found in all \acrshortpl{ka}, and that safety/security and non-technical areas in particular have received a lot of attention in the literature.

\textcite{Nascimento2020} performed an \acrshort{slr} of challenges and practices in \acrshort{se} for \acrshort{ml}. The challenges and practices are mapped to the \acrshortpl{ka} of the SWEBOK.
The authors found that studies from laboratory environments are mainly concerned with building and testing models, which could indicate that more literature produced by industry practitioners should be studied when looking to review the state of the art and state of practice of deploying models to production.
This is further illustrated when the authors highlight that deployment is one of the areas with the fewest suggested practices in published literature.

\textcite{Lwakatare2020} conducted an \acrshort{slr} on  \acrshort{se} for \acrshort{ml} using a two-dimensional scheme, categorizing challenges according to quality attributes (adaptability, scalability, privacy and safety) and \acrshort{ml} development workflow step (data acquisition, training, evaluation, deployment).
The authors also identify solutions to some of the challenges, but note that areas such as privacy and safety are missing solutions, as well as areas such as evaluation and deployment.
The exclusion of \acrshort{gl} is mentioned as a threat to validity of the study.

\textcite{Lwakatare2020a} did an exploratory case study on \acrshort{cd} for \acrshort{ml}.
The authors performed an \acrshort{mlr} to create a five-step process improvement conceptual model for the \acrshort{ml} deployment process. 
The results were validated by conducting a focus group with ten \acrshort{ml} practitioners at a telecommunications company.

\textcite{Serban2020} identified 29 best practices of \acrshort{se} for \acrshort{ml} through conducting an \acrshort{mlr}, before conducting a practitioner survey to investigate adoption and impact of the practices.
The authors perform extensive quantitative analysis of the survey responses, investigating the relationships between practice and effect, correlation between practices, and the importance of each practice.

\textcite{Karamitsos2020} perform an \acrshort{slr} in order to propose a CI/CD pipeline for ML systems, providing suggestions for practices and tools to facilitate the pipeline.
However, the paper lacks rigor, reporting no details of the search, study selection, data extraction or data synthesis procedures, and providing no discussion of the results or threats to validity.

\textcite{John2021} performed an \acrshort{mlr} of the SE lifecycle of ML models, outlining challenges and best practices in each step of the lifecycle.

\textcite{Giray2021} conducted an \acrshort{slr} of \acrshort{se} for \acrshort{ml}, where challenges and solutions were grouped by areas suggested in SWEBOK, similarly to \cite{Kumeno2020} and \cite{Nascimento2020}, while including an even greater number of papers than any previous SLR in the field.

\textcite{Lorenzoni2021} conducted an \acrshort{slr} of \acrshort{ml} model development, identifying phases, techniques, gaps and trends in the \acrshort{ml} development lifecycle.
The deployment aspect is only briefly mentioned.

\textcite{MartinezFernandez2021} performed an extensive \acrshort{sms} of SE4AI, investigating reported SE approaches and challenges for ML.

\textcite{Serban2021} studied software architecture for machine learning, conducting an \acrshort{slr} and practitioner interviews to find architectural challenges and solutions in ML, while validating the findings with a survey.

\textcite{John2021a} performed an \acrshort{slr} and \acrshort{glr} on the adoption of MLOps practices and how companies evolve through different stages of adoption.
A framework is produced, allowing the characterization of a company's MLOps adoption maturity and providing steps on how to improve adoption, similarly to \cite{Lwakatare2020a}.

A common characteristic of previous work is that it tends to be quite broad in scope, covering many areas of \acrshort{se} for \acrshort{ml}.
This study will make the following novel contributions to the research.
\begin{itemize}
    \item Focus specifically on how ML models are deployed, including the aspects of serving and monitoring, which have largely been overlooked in previous work
    \item Investigate deployment tooling and infrastructure, and identify reported gaps and challenges
    \item Include \acrshort{gl} as well as \acrshort{wl}
\end{itemize}

\begin{table}[h]
    \centering
    \begin{tabular}{l c c c c}
        Study & D & \acrshort{gl} & TI & SE4ML \\
        \hline
        \cite{Shahin2017} & X & & X & \\
        \cite{Rodriguez2017} & X & & & \\
        \cite{Baier2019} & X & & & X \\
        \cite{Paleyes2020} & X & & & X \\
        \cite{Kumeno2020} & & & & X \\
        \cite{Nascimento2020} & & & & X \\
        \cite{Lwakatare2020} & & & & X \\
        \cite{Lwakatare2020a} & & X & & X \\
        \cite{Serban2020} & & X & & X \\
        \cite{Karamitsos2020} & X & & X & X \\
        \cite{John2021} & X & X & & X \\
        \cite{Giray2021} & & & & X \\
        \cite{Lorenzoni2021} & & & & X \\
        \cite{MartinezFernandez2021} & & & & X \\
        \cite{Serban2021} & & & & X \\
        \cite{John2021a} & & X & & X \\
        This study & X & X & X & X \\
    \end{tabular}
    \caption{Comparison of related work with this study. D --- in-depth discussion of how models are deployed; GL --- reviews GL; TI --- investigates tooling and infrastructure issues; SE4ML --- concerns the application of SE to ML.}
    \label{tab:related_work_comparison}
\end{table}

As can be seen from \cref{tab:related_work_comparison}, \cite{Karamitsos2020} and \cite{John2021} appear to be the closest competitors to this study.
However, there are some key points differentiating this study from each of them.
First, \cite{John2021} 

\todo[inline]{Maybe write something about why karamitsos is a very low-quality study}