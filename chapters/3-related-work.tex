\chapter{Related Work}
Multiple systematic literature studies have already been conducted on the field of Software Engineering for AI/ML.

\textcite{Kumeno2020} performed a systematic literature review (SLR) of the field, where challenges in SE4ML were mapped to the twelve knowledge areas (KAs) of the SWEBOK, which span both software and engineering perspectives of SE.

\textcite{Nascimento2020} performed an SLR mapping practices in SE4ML to the KAs of the SWEBOK.
The authors found that studies from laboratory environments are mainly concerned with building and testing models, which could indicate that more literature produced by industry practitioners should be studied when looking to review the state of the art and state of practice of deploying models to production.
This is further illustrated when the authors highlight that deployment is one of the areas with the fewest suggested practices in published literature.

\textcite{Lwakatare2020} conducted an SLR on SE4ML using a two-dimensional approach, categorizing  challenges according to both  quality  attributes  (adaptability, scalability,privacy and safety) and ML development workflow step (data acquisition, training, evaluation, deployment).
The authors mentioned the exclusion of grey  literature (GL) as a potential weakness of their study.

\textcite{John2021} performed a Multivocal Literature Review (MLR) of the SE lifecycle of ML models, outlining challenges and best practices in each step of the lifecycle.

\textcite{Giray2021} conducted an SLR of SE for ML, where challenges and solutions were grouped by areas suggested in SWEBOK, similarly to \cite{Kumeno2020} and \cite{Nascimento2020}, while including an even greater number of papers than any previous SLR in the field.

\textcite{MartinezFernandez2021} performed an extensive systematic mapping study (SM) of SE4AI.

A common characteristic of previous work is that it tends to be quite broad in scope, covering many steps of the SE4ML process.