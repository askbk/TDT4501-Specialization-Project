\chapter{Discussion}
\label{ch:discussion}
\cref{ch:discussion:comparison_to_related_work} will discuss how this study's results compare with those of previous related work.
\cref{ch:discussion:limitations} will discuss limitations of this study, as well as steps taken to mitigate some of the threats to validity.

\section{Comparison to Related Work}
\label{ch:discussion:comparison_to_related_work}
The results of this study should shine some new light on operationalization of ML.
First off, it is the only \acrshort{slr}, as far as the author knows, that investigates operationalization of \acrshort{ml} models with respect to tooling and infrastructure.
\cref{ch:research_results:rq1_how} describes how ML models are operationalized in the literature, with an emphasis on the infrastructure aspect which is completely novel to the knowledge of the author.
\cref{ch:research_results:rq3_tools_infrastructure} gives an overview of the specific software used to operationalize ML models, which has not been done in previous SLRs either.
Secondly, it is also the only \acrshort{slr} on the subject to identify reported suggestions for further research, which should assist in guiding researchers in directions of future studies.

\section{Limitations}
\label{ch:discussion:limitations}
\subsection{Threats to Internal Validity}
Being an \acrshort{mlr}, there are certain inherent threats to validity.
There is the possibility that not all relevant literature is found or included.
To mitigate this, the start set included papers spanning multiple years (2017-2021), venues (IEEE, ACM, Sciendo, Frontiers, Sciencedirect, Zenodo, IGI Global, Wiley, MDPI, Springer) and authors.
However, as the snowballing procedure was only limited to a single iteration of forward and backward snowballing, instead of continuing until no new studies are found, it is quite probable that this review is not exhaustive.

As the review was conducted by a single person, there is the probability of bias and mistakes in all stages of the review process.
The application of selection criteria and quality assessment criteria may have been inconsistent or biased, information may have been overlooked during the data extraction process, and the analysis could contain unfounded or biased conclusions.
Furthermore, qualitative studies are more closely tied to the researchers' background, identity, assumptions and beliefs than quantitative studies according to \textcite{Oates2005}.
To counteract theses effects, the each stage of the review could have been either been conducted in parallel by another researcher.
In addition, a sensitivity analysis could have been performed by synthesizing data from a random subset of the reviewed literature to determine if the conclusion would differ significantly.

\subsection{Threats to External Validity}
This study suffers from some unmitigated threats to external validity, factors which may render the conclusions of a study inapplicable outside the context of the study.
As the field of MLOps is still young and rapidly evolving, there is the strong possibility of information reported in published research being outdated shortly after publication.
This review includes studies published as early as 2017, meaning that the conclusions drawn may be based on outdated data and therefore be of limited value.
To mitigate this, the selection criteria for recency could have been made stricter, e.g. only including studies published in 2020 and later.
However, it is not known if any of the reviewed studies are outdated or if the quality of is review would have improved by excluding some of the older papers.

Additionally, the study potentially misses out on practitioner knowledge because GL is not included in the review.
While it is more challenging to assess the quality and reliability of GL, it could provide valuable practitioner insights, especially in a practitioner-oriented field like SE \cite{Garousi2016}.
A GLR was not conducted in this study because of time constraints, but would be a natural direction to follow in future research.

