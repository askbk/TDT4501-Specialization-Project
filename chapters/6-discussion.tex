\chapter{Discussion}
\label{ch:discussion}
\cref{ch:discussion:comparison_to_related_work} will discuss how this study's results compare with those of previous related work.
\cref{ch:discussion:limitations} will discuss limitations of this study, as well as steps taken to mitigate some of the threats to validity.

\section{Comparison to Related Work}
\label{ch:discussion:comparison_to_related_work}
This study has contributed the following new knowledge.
\begin{itemize}
    \item How \acrshort{ml} models are operationalized with respect to tools and infrastructure. This includes various techniques for packaging and integration, deployment, serving and monitoring, which to the author's best knowledge have not been investigated in any previous \acrshort{slr}.
    \item An overview of what tools have reportedly been used to operationalize \acrshort{ml} models in the literature.
    \item An overview of areas for further research into \acrshort{ml} operationalization as suggested by the literature.
\end{itemize}

The results of the review show that researchers should work on two areas.
First, performance and scalability in prediction serving, possibly using a combination of existing techniques for batching, caching and dynamic model selection.
Second, edge environments in general were reported to have general open problems in deployment and monitoring.

The results of the study could be of value to industry practitioners as a reference for various techniques and patterns for operationalizing \acrshort{ml}.
In addition, the study could also serve as a starting point for understanding what tools are used in different stages and contexts of operationalization, as well as what challenges may be expected when operationalizing \acrshort{ml} models.

\section{Limitations}
\label{ch:discussion:limitations}
\subsection{Threats to Internal Validity}
Being an \acrshort{slr}, there are certain inherent threats to validity.
There is the possibility that not all relevant literature is found or included.
To mitigate this, the start set included papers spanning multiple years (2017-2021), venues (IEEE, ACM, Sciendo, Frontiers, Sciencedirect, Zenodo, IGI Global, Wiley, MDPI, Springer) and authors.
However, as the snowballing procedure was only limited to a single iteration of forward and backward snowballing, instead of continuing until no new studies are found, it is quite probable that this review is not exhaustive.

As the review was conducted by a single (and inexperienced) person, there is the probability of bias and mistakes in all stages of the review process.
The application of selection criteria and quality assessment criteria may have been inconsistent or biased, information may have been overlooked during the data extraction process, and the analysis could contain unfounded or biased conclusions.
In particular, as a simplified thematic analysis procedure was followed, the data synthesis was not as rigorous as suggested by \cite{Lochmiller2021}.
Furthermore, qualitative studies in general are more closely tied to the researchers' background, identity, assumptions and beliefs than quantitative studies according to \cite{Oates2005}.
To counteract some of these personal biases, possible issues were discussed with the supervisor.
A more robust approach, however, would have had at least one other researcher conducting each step of the review in parallel, using tests such as Fleiss' Kappa to determine level of aggreement between researchers.
In addition, a sensitivity analysis could have been performed by synthesizing data from a random subset of the reviewed literature to determine if the conclusion would differ significantly.

\subsection{Threats to External Validity}
This study suffers from some unmitigated threats to external validity, factors which may render the conclusions of a study inapplicable outside the context of the study.
As the field of MLOps is still young and rapidly evolving, there is the strong possibility of information reported in published research being outdated shortly after publication.
This review includes studies published as early as 2017, meaning that the conclusions drawn may be based on outdated data and therefore be of limited value.
To mitigate this, the selection criteria for recency could have been made stricter, e.g. only including studies published in 2020 and later.
However, it is not known if any of the reviewed studies are outdated or if the quality of this review would have improved by excluding some of the older papers.

Additionally, the study potentially misses out on practitioner knowledge because \acrshort{gl} is not included in the review.
While it is more challenging to assess the quality and reliability of \acrshort{gl}, it could provide valuable practitioner insights, especially in a practitioner-oriented field like \acrshort{se} \cite{Garousi2016}.
A \acrshort{glr} was not conducted in this study because of time constraints, but would be a natural direction to follow in future research.

