\chapter{Research Design and Implementation}
\label{ch:research_design_and_implementation}
This chapter will discuss the motivation, design and implementation of the SLR.
The chapter is structured as follows.
In \cref{sec:research_motivation} the impact and usefulness of the study is argued.
\cref{sec:research_questions} presents the research questions of the study.
\cref{sec:research_method} outlines the research method and design of the study.
Finally, \cref{sec:research_implementation} accounts for how the study was conducted.

\section{Research Motivation}
\label{sec:research_motivation}
As outlined in \cref{ch:background}, industry adoption of ML is still in its early stages and rapidly growing.
The SE aspect of ML is an active field of research, with the vast majority of work having been done only in the past five years.
As reported by earlier literature studies in the field (see \cref{ch:related_work}), deployment of ML is an area which presents real challenges to practitioners.
Tackling deployment challenges requires adopting good practices as well as utilizing suitable tooling.
Much of the earlier work has largely had a broad scope, with goals of mapping out broad SE challenges and practices for ML.
Further, earlier work has had a focus on reviewing the academic literature, ignoring the vast body of practitioner knowledge found in the GL.
To summarize, this study is motivated by three factors.
\begin{itemize}
    \item Researching ML deployment specifically, and not as part of a broader study of SE for ML.
    \item Including more practitioner knowledge in the form of GL.
    \item Putting more focus on tooling by identifying what needs are fulfilled by current tooling and what is missing from tooling.
\end{itemize}

\section{Research Questions}
\label{sec:research_questions}
The overarching research question of this study is "\textbf{What is the state of the art and state of the practice in ML deployment?}".
The focus will be on tooling and trying to identify feature gaps reported in studies where model deployment is discussed.
To support the main research question, four subquestions have been formulated:
\begin{itemize}
    \item \textbf{RQ1: How is ML model deployment handled in real-world applications?}
    \item \textbf{RQ2: What are the main challenges and pain points in ML model deployment?}
    \item \textbf{RQ3: What tools are used to deploy ML models?}
    \item \textbf{RQ4: Are there any feature gaps in the tooling used to deploy ML models?}
\end{itemize}


\section{Research Method and Design}
\label{sec:research_method}
This study is a multivocal literature study (MLR) largely based on the guidelines in \textcite{Kitchenham07guidelinesfor} and \textcite{Garousi2019}.

\subsection{Search Strategy}
\subsubsection{White Literature}
Initial string-based searches in digital libraries failed to produce adequately relevant articles for the topic at hand, thus motivating the use of snowballing as an alternative approach for identifying candidate papers.
The snowballing procedure followed the process proposed by \textcite{Wohlin2014}, with the exception of limiting forward and backward snowballing to a single iteration each.

\subsubsection{Grey Literature}
Pilot searches using Google revealed that adequately relevant and recent results were found using the search string "mlops deployment".
Based on the outlined GL search guidelines in \cite{Garousi2016}, the Top N-approach was selected with the search string "mlops deployment" and $N=100$.

\subsection{Study Selection}
\subsubsection{White Literature}
Inclusion criteria for WL:
\begin{itemize}
    \item Written in English
    \item Published in a peer-reviewed journal, conference or workshop
    \item Published after 2010
    \item Discusses one of the following aspects of machine learning deployment: challenges, solutions, tooling, processes, requirements
    \item Available online
\end{itemize}

Exclusion criteria for WL:
\begin{itemize}
    \item One of the inclusion criteria is not satisfied
\end{itemize}

\subsubsection{Grey Literature}
Inclusion criteria for GL:
\begin{itemize}
    \item Written in English
    \item Published after 2010
    \item Discusses one of the following aspects of machine learning deployment: challenges, solutions, tooling, processes, requirements
    \item Available online
\end{itemize}

Exclusion criteria for GL:
\begin{itemize}
    \item One of the inclusion criteria is not satisfied
\end{itemize}

\subsection{Study Quality Assessment}
\subsection{Data Extraction}
In order to answer the research questions, sufficient data must be gathered from the literature.
Thus, the following data extraction form was designed:

\begin{itemize}
    \item Source type (WL/GL)
    \item Publication date
    \item Process
    \begin{itemize}
        \item What was the deployment process?
    \end{itemize}
    \item Tools
    \begin{itemize}
        \item What tools were used for deployment?
        \item What tooling features were used for deployment?
    \end{itemize}
    \item Challenges
    \begin{itemize}
        \item Is any deployment challenge reported?
        \item What was the challenge?
    \end{itemize}
    \item Solutions
    \begin{itemize}
        \item What challenge is addressed?
        \item What is the solution?
        \item Is the solution merely theoretical (proposed), or do the authors implement it?
        \item If the solution was implemented, what was the outcome?
    \end{itemize}
    \item Requirements
    \begin{itemize}
        \item What functional requirements do the authors state for deployment tooling?
        \item What non-functional requirements do the authors state for deployment tooling?
        \item What is the reasoning for the stated requirements?
    \end{itemize}
    \item Gaps
    \item \begin{itemize}
        \item Is any shortcoming in deployment tooling either implicitly or excplicitly reported?
        \item What are the shortcomings?
        \item How was the shortcoming inferred from the literature (by me)?
    \end{itemize}
    \item Other
\end{itemize}

\subsection{Data Analysis}
\subsection{Data Synthesis}

\section{Research Implementation}
\label{sec:research_implementation}